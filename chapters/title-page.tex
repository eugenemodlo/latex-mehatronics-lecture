\begin{titlepage}
 \begin{center}
  \textbf{МІНІСТЕРСТВО ОСВІТИ І НАУКИ УКРАЇНИ} \\
  \textbf{ДЕРЖАВНИЙ УНІВЕРСИТЕТ ЕКОНОМІКИ І ТЕХНОЛОГІЙ} \\
  \textbf{Навчально-науковий технологічний інститут} \\
  \textbf{Кафедра електричної інженерії та автоматизації}
 
\vspace{5.0em}

\large { \textbf{ОСНОВИ МЕХАТРОНІКИ}}

Конспект лекцій для підготовки студентів денної та заочної форми навчання спеціальностей: 

141 Електроенергетика, електротехніка та електромеханіка,

151 Автоматизація, та компютерно-інтегровані технології.

\vspace{2cm}

\begin{tabular}{p{0.25\linewidth}l}

& РЕКОМЕНДОВАНО
\\& На засіданні кафедри електричної
\\& інженерії та автоматизації
\\& Протокол \No \hrf{1cm} від \hrf{3cm}
	
\vspace{1.5cm}
	
\\& ЗАТВЕРДЖЕНО
\\& Науково-методичною радою
\\& Державного університету 
\\&економіки і технологій
\\& Протокол \No \hrf{1cm} від \hrf{3cm}
	
\end{tabular}

\vspace{2cm}

Кривий Ріг -- 2023
\end{center}

Конспект лекцій із дисципліни ``Основи мехатроніки'' для підготовки студентів денної та заочної форми навчання спеціальностей: 141 Електроенергетика, електротехніка та електромеханіка та 151 Автоматизація, та компютерно-інтегровані технології. / Державний університет економіки і технологій, Навчально-науковий технологічний інститут, кафедра електричної інженерії та автоматизації; уклад. Є. МОДЛО, рец. О. УЧИТЕЛЬ., Кривий Ріг 2023, \pageref{LastPage}с.

\vspace{2cm}
Укладач: Євгеній МОДЛО, в.о. завідувача кафедри електричної інженерії та автоматизації, к.п.н.

\vspace{2cm}
Рецензент: Олександр УЧИТЕЛЬ, д.т.н., професор кафедри електричної інженерії та автоматизації

\vspace{2cm}
 Відповідальний за випуск: Євгеній Модло, в.о. завідувача кафедри електричної інженерії та автоматизації, к.п.н.

\vspace{2cm} 
Конспект лекцій із дисципліни ``Основи мехатроніки'' для підготовки студентів денної та заочної форми навчання спеціальностей: 141 Електроенергетика, електротехніка та електромеханіка та 151 Автоматизація, та компютерно-інтегровані технології, денної та заочної форм навчання розроблено у відповідності до навчального плану з метою узагальнення матеріалу, що викладається на лекціях із дисципліни ``Основи мехатроніки''.
\end{titlepage}